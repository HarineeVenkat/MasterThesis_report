\chapter{Main Part of the Thesis}
\section{\label{sec:intro}Introduction}
This work demonstrates a sustainable nuclear fusion reaction of hydrogen 
using a clay flower port as a reactor vessel. Our novel approach uses
a ``charge mirror" that reduces the electromagnetic repulsion between 
nuclei enough to allow fusion initiation at room temperature.
The device can also be used as a secure error-free transgalactic communications
pipe with zero latency and near infinite bandwidth.   

\subsection{\label{sec:intro:further}Further Introduction}
This work is completely radical.  We really don't need to cite any 
references.  However, we will cite ourselves~\cite{epr35,feyn54,feynman59} 
just to increase our citation record.  Here are some 
more~\cite{einstein67,bell24,bell34}.

\subsubsection{\label{sec:intro:further:more}More Introduction}
This work demonstrates a sustainable nuclear fusion reaction of hydrogen 
using a clay flower port as a reactor vessel. Our novel approach uses
a ``charge mirror" that reduces the electromagnetic repulsion between 
nuclei enough to allow fusion initiation at room temperature.
The device can also be used as a secure error-free transgalactic communications
pipe with zero latency and near infinite bandwidth.   

\subsubsection{\label{sec:intro:further:evenmore}Even More Introduction}
This work demonstrates a sustainable nuclear fusion reaction of hydrogen 
using a clay flower port as a reactor vessel. Our novel approach uses
a ``charge mirror" that reduces the electromagnetic repulsion between 
nuclei enough to allow fusion initiation at room temperature.
The device can also be used as a secure error-free transgalactic communications
pipe with zero latency and near infinite bandwidth.   

\subsection{\label{sec:intro:ending}Ending Introduction}
This work demonstrates a sustainable nuclear fusion reaction of hydrogen 
using a clay flower port as a reactor vessel. Our novel approach uses
a ``charge mirror" that reduces the electromagnetic repulsion between 
nuclei enough to allow fusion initiation at room temperature.
The device can also be used as a secure error-free transgalactic communications
pipe with zero latency and near infinite bandwidth.   

This work demonstrates a sustainable nuclear fusion reaction of hydrogen 
using a clay flower port as a reactor vessel. Our novel approach uses
a ``charge mirror" that reduces the electromagnetic repulsion between 
nuclei enough to allow fusion initiation at room temperature.
The device can also be used as a secure error-free transgalactic communications
pipe with zero latency and near infinite bandwidth.   

See Sec.~\ref{sec:intro}, Sec.~\ref{sec:intro:further}, and 
Sec.~\ref{sec:intro:further:evenmore} for details.

\section{\label{sec:another}Another section}
This work demonstrates a sustainable nuclear fusion reaction of hydrogen 
using a clay flower port as a reactor vessel. Our novel approach uses
a ``charge mirror" that reduces the electromagnetic repulsion between 
nuclei enough to allow fusion initiation at room temperature.
The device can also be used as a secure error-free transgalactic communications
pipe with zero latency and near infinite bandwidth.   

Figure~\ref{fig:claypot} shows the experimental apparatus.
The center asterisk shows the location of the fusion reaction.

\begin{figure}
\begin{center}
\includegraphics*[width=3.3in]{images/figure}
\caption{\label{fig:claypot}
Schematic diagram of the experimental apparatus 
for nuclear fusion in a clay flower pot.}
\end{center}
\end{figure}

This work demonstrates a sustainable nuclear fusion reaction of hydrogen 
using a clay flower port as a reactor vessel. Our novel approach uses
a ``charge mirror" that reduces the electromagnetic repulsion between 
nuclei enough to allow fusion initiation at room temperature.
The device can also be used as a secure error-free transgalactic communications
pipe with zero latency and near infinite bandwidth.   

\section{\label{sec:theory}Theory}
The theory can't possibly be understood by anyone but us.  
Nevertheless, we give here the key equation
\begin{equation}
\label{eqn:energy}
E^2=(pc)^2+(mc^2)^2\,.
\end{equation}
Obviously, Eq.~(\ref{eqn:energy}) says it all.

This work demonstrates a sustainable nuclear fusion reaction of hydrogen 
using a clay flower port as a reactor vessel. Our novel approach uses
a ``charge mirror" that reduces the electromagnetic repulsion between 
nuclei enough to allow fusion initiation at room temperature.
The device can also be used as a secure error-free transgalactic communications
pipe with zero latency and near infinite bandwidth.   

This work demonstrates a sustainable nuclear fusion reaction of hydrogen 
using a clay flower port as a reactor vessel. Our novel approach uses
a ``charge mirror" that reduces the electromagnetic repulsion between 
nuclei enough to allow fusion initiation at room temperature.
The device can also be used as a secure error-free transgalactic communications
pipe with zero latency and near infinite bandwidth.   

This work demonstrates a sustainable nuclear fusion reaction of hydrogen 
using a clay flower port as a reactor vessel. Our novel approach uses
a ``charge mirror" that reduces the electromagnetic repulsion between 
nuclei enough to allow fusion initiation at room temperature.
The device can also be used as a secure error-free transgalactic communications
pipe with zero latency and near infinite bandwidth.   

\section{\label{sec:conclusion}Conclusion}
This work demonstrates a sustainable nuclear fusion reaction of hydrogen 
using a clay flower port as a reactor vessel. Our novel approach uses
a ``charge mirror" that reduces the electromagnetic repulsion between 
nuclei enough to allow fusion initiation at room temperature.
The device can also be used as a secure error-free transgalactic communications
pipe with zero latency and near infinite bandwidth.   
