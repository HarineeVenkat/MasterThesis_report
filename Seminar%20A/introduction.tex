\chapter{Introduction}
The Thesis start here\dots

\subsection{Template and Software for writing your thesis}

We would like to encourage you to use LaTeX for your written report. To support this, we prepared a template file you can use as a blueprint and fill in your texts. Download it below.

For figures, diagrams and other images you should always use a vector format. Useful software is mainly:
\begin{itemize}
\item[] dia from http://www.gnome.org/projects/dia/
\item[] inkscape from http://www.inkscape.org/
\item[] gimp from http://www.gimp.org
\item[] graphviz from http://www.graphwiz.org
\item[] gnuplot
\end{itemize}
These programs are available on almost all OS platforms and can
save eps files. For simple figures, you can also use xfig, but
it's only available on Unix/Linux. With graphviz you can
visualise graph structures, e.g., internal data structures.

\begin{itemize}
\item[] jabref http://jabref.sourceforge.net (organise your
        bibliography)
\end{itemize}

\subsection{Important Hints for writing your thesis}
(included topic: plagiarism)

\begin{description}
\item{Citings:} Citings of primary or secondary literature:
if shorter than 3 lines: in the running text, in quotes;
if longer than 3 lines: as a paragraph, indented.
\item{References:} when?
After each citing from primary or secondary literature: (cite page\_number)
\begin{itemize}
\item Example: "This is copied from text one on page fourteen" ([1], p.14).
After each literally or correspondingly assumed contents of secondary literature (paraphrase)
\item Example: This conclusion appears quite similar in text one [1].
\end{itemize}
\item{Plagiarism:}
{\bf It is not acceptable} to use assumed material without reference! (this includes images)
Also, read this text: http://www.informatik.tu-darmstadt.de/Plagiarism (there is at least one link to an English page also)
\item{Wikipedia:} the wikipedia in general is not a reviewed encyclopaedia! Be careful with the contents (correct it when wrong - contribute to the community!) and do not cite it extensively!
\end{description}

\subsection{Additional hints for your thesis:}

\begin{itemize}
\item Please check if your LaTeX source can be {\bf compiled on our  computers!}
\item Unix/Linux file names are {\bf case sensitive}, a frequent mistake are wrong filenames when exporting files from a PC!
(e.g. picture.eps is not the same as picture.EPS or Picture.eps)
\item We can support a low number of color images inside your thesis, the according pages can be printed here on our color laser.
\item Please consider these hints when preparing your thesis! (list is incomplete and preliminary)
\item Last but not least: make backups of your data if you work at home!
You can (should) always have a recent copy, not older than 2 days, of all your work stuff on our computers here; just transfer your data by sftp or a USB stick to your MES account!
\item see the {\bf check list} on the web page for your hand-in
\end{itemize}

\subsection{About Coding and Describing Code}

Please consider that comments and documentation is more than 50\%
of what's needed for good code! Uncommented and undocumented code
is almost unusable later and thus almost worthless!

 Writing your Code:

\begin{itemize}
\item please write clear and readable!
\item please comment your code (in Englisch, without umlauts)!
\item for Java code, use JavaDoc as standard, for VHDL VHDLDoc. 
\end{itemize}

For the documentation of Java code:

\begin{itemize}
\item first, give a general overview of the programm and class structure, and the concept of the used data structures and program flow
\item make a subsection for each (major) class
\item describe general function, variables, methods, functions, parameters of functions etc. within each class; use tables and images to support your text!
\item describe all algorithms in detail, as well as implementation details
\item show examples of what your program does (test cases and data) 
\end{itemize}

For the documentation of VHDL/Verilog code:
\begin{itemize}
\item give and describe a top-level view of your design, including the file structure
\item make a subsection for each (major) module
  BTW, mention the filenames where to find the module)
\item describe sub-modules, structure (e.g., register-transfer-idea), interfaces (I/Os), specific timing, other conditions
\item describe all algorithms in detail, as well as implementation details
\item show examples of what your hardware does (test cases and data)
\item write and describe testbenches!!!
\item generally: {\bf draw your own figures}, use automatically generated block diagrams (from design tools) only in special cases!
\end{itemize}


\subsection{\TeX \,Example}
\noindent
A boxed equation:
\begin{equation}
\boxed{\int\limits_{-\infty}^{\infty}\delta(x)\mathrm{d}x=1}
\end{equation}
and an unboxed one:
\begin{equation}
U=I\varrho\frac{l}{A}
\end{equation}

