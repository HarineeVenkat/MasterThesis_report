\chapter{"Uberschriften}
\label{cha:uberschriften}

F"ur solch eine Arbeit wird der Text in diverse Abschnitte, Unterabschnitte und evtl. auch noch Unterunterabschnitte unterteilt. Diese m"ussen dann mit der entsprechenden "Uberschrift gekennzeichnet werden. Die Dokumentklasse parkskip kennt hier die Unterscheidung in chapter, section, subsection, subsubsection und paragraph.

Der Befehl {\textbackslash chapter\{"Uberschrift1\}} beginnt ein neues Kapitel, erzeugt eine Kapitel"uberschrift und tr"agt diese ins Inhaltsverzeichnis ein. So ist der oben genante Kapitel mit dem Befehl {\textbackslash chapter\{"Uberschriften\}} als chapter definiert.

Optional kann man mit Hilfe des Befehls {\textbackslash chapter[Kurzform]\{"Uberschrift\}} eine Kurzform f"ur den Kapitelname angeben. Die Kurzform wird dann anstelle der "Uberschrift ins Inhaltsverzeichnis eingetragen.
\paragraph{Beispiel}
\label{par:beispiel}

{\textbackslash chapter[Anf"ange (1920)]\{Anf"ange der modernen Science--Fiction--Literatur (1920)\}}. Im Inhaltsverzeichnis erscheint nur "Anf"ange (1920)".

\section{"Uberschrift2}
\label{sec:uberschrift2}

Mit dem Befehl {\textbackslash section\{"Uberschrift2\}} wird einen neuen Abschnitt des Dokuments auf der section-Ebene erzeugt. Die zugeh"orige "Uberschrift wird definiert und ins Inhaltsverzeichnis eingetragen. Wenn eine Kurzform erw"unscht ist, kann sie auch mit dem entsprechenden Befehl erzeugt werden (Siehe Beispiel unter Kapitel\ref{par:beispiel}).

\subsection{"Uberschrift3}
\label{sec:uberschrift3}

In der subsection-Ebene, wird ebenfalls die zugeh"orige "Uberschrift erzeugt und ins Inhaltsverzeichnis eingetragen. Hier ist auch ebenfalls m"oglich eine Kurzform angegeben. Der Befehl lautet folgenderma"sen {\textbackslash subsection\{"Uberschrift3\}}.

\subsubsection{"Uberschrift4 und Paragraph}
\label{sec:uberschrift4}

Der Befehle dazu lauten {\textbackslash subsubsection\{"Uberschrift4\}} bzw. {\textbackslash paragraph\{"Paragraph\}}. Die "Uberschriften f"ur diese Ebenen erfolgen ohne Nummerierung und werden nicht im Inhaltsverzeichnis aufgenommen. Das o.g. Beispiel wurde z. B. als Paragraph definiert.





